\documentclass[dvips]{article}
\usepackage[pdftex]{graphicx}
\usepackage{url}
\usepackage{caption}
\usepackage{multicol}
\usepackage{layout}
\usepackage[margin=1in]{geometry}

%\setlength{\voffset}{-0.75in}
%\setlength{\headsep}{5pt} 
 
\begin{document}
\sloppy
\title{Performance Review 4}
 
\author{Peter Macgregor, s1126155}
 
\maketitle

\section{Your Contribution to the Group}
Once again, I've been working on vision throughout the last few weeks. I created a GUI,
included functionality for saving settings, and improved the algorithms for tracking the robots.
I also included other functionality as requested by the rest of the team. I was then available
during testing to ensure that vision was working correctly and to teach the team to set it up.

\section{High Points of Your Contribution}
\begin{itemize}
  \item Creating GUI for running and fine-tuning vision.
  \item Added the ability to save vision settings.
  \item Improved robot tracking.
\end{itemize}

\section{Points Awarded to Each Member (out of 5)}
\begin{itemize}
\begin{multicols}{2}
\item Peter Macgregor - 4
\item Patricia Dragan - 1.5
\item Egidijus Skinkys - 3.5
\item Andrew Leith - 4
\item Nursultan Askarbekuly - 3
\item Peak Limpiti - 4.5
\item Zhang Chen - 4
\item Aris Tsialos - 2
\item Kuba Kaszyk - 4
\end{multicols}
\end{itemize}

\section{Additional Information}
The code for running the GUI and saving state was roughly based on code from last years group 4.
The robot tracking now relies heavily on detecting the green plates rather than just the color
of the 'i's. The plates are much easier to detect well using vision. I also wrote a wiki
article on setting up vision since it requires setting up in a certain way due to the way I
track the robots.



\end{document}

