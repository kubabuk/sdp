\documentclass[dvips]{article}
\usepackage[pdftex]{graphicx}
\usepackage{url}
\usepackage{caption}
\usepackage{multicol}
\usepackage{layout}
\usepackage[margin=1in]{geometry}

%\setlength{\voffset}{-0.75in}
%\setlength{\headsep}{5pt} 
 
\begin{document}
\sloppy
\begin{figure}
\centering
\includegraphics[width=90mm]{logo.jpg}
\label{overflow}
\end{figure}

\title{Milestone 1 Group 6 Summary}
 
\maketitle
\begin{itemize}
\item Peter Macgregor - 4 - Peter worked primarily on vision. While the vision team having problems with installing our first choice library in Java, Peter set up a working backup in Python and wrote a ball recognizer. He contributed to the design of the robot by adding a second sensor and rewriting the code so that the robot could run both clockwise and anti-clockwise. Peter also helped with testing before the milestone and fixing bugs.
\item Patricia Dragan - 3 - Patricia contributed to the set up of the video feed on DICE. She also worked with the design team researching possible builds for the robots, and decided on the initial sensor placement. 
\item Egidijus Skinkys - 2 - Egidijus helped set up the initial firmware on the NXTs and software in Eclipse. Along with Andrew, he measured the light intensity at different points on the pitch. He also helped with the set up of communications, working out MAC addresses for the bricks. He contributed less than the rest of the team due to illness.
\item Andrew Leith - 3 - Andrew, along with Egidijus set up the initial firmware on the NXTs. He then worked with Aris on the halting algorithm for the robot. After a lot of time researching and battling with DICE, he was able to set up bluetooth communications between the NXTs and the computers. He helped a lot with testing, and rewriting key parts of the Algorithm.
\item Nursultan Askarbekuly - 4 - Nur put a lot of time into researching and building our robot. He built multiple designs and was always prepared to respond to the requests of the rest of the team. Many decisions in the robot used in Milestone 1 were direct effects of his prototypes. 
\item Peak Limpiti - 3 - Peak worked on the engineering design of the robots along with Nur. He researched circular-vertical actuators, and worked on making the design of the robot more compact. 
\item Zhang Chen - 3 - Zhang Chen (Roy) contributed the initial code for the movement of the robot, as well as the first draft of the algorithm for completing Milestone 1. Independently, he researched simulators suitable to represent football robots, and began initial setup in Webots.
\item Aris Tsialos - 4 - Aris was responsible for many high level designs that have yet to be implemented - he designed the pipleine His insight was key in relevant changes made during the testing phase. Along with Andrew, he designed the halting algorithm for the robot, necessary to complete Milestone 1. He also worked closely with the team leader and helped with organisational tasks, maintaining the repository as well as the wiki.
\item Kuba Kaszyk - 4 - Kuba's primary task as team leader was organisation and communications between team members. He set up meetings, organized the wiki, created timesheets and templates for everyone to better communicate and document. Kuba also spent a lot of time setting up the initial video feed with Java. After he was able to make it work, he began writing a ball recognizer in Java. He spent some time working with Roy and Patricia on the initial robot movements. Along with Aris, Andrew and Peter, he worked a lot with final testing and bug fixes.
\end{itemize}


\end{document}

